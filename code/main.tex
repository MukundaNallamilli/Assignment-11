\documentclass{beamer}

\usepackage{amssymb}
\usepackage{amsfonts}
\usepackage{amsmath}
\usepackage{amsthm}
\usepackage{setspace}
\usepackage{longtable}
\usepackage{graphicx}
\usepackage{mathtools}
\usepackage{color}
\usepackage{array}
\usepackage{calc} 
\usepackage{bm}
\usepackage{caption}
\usepackage{float}

\usetheme{CambridgeUS}
\useoutertheme{infolines}
%numbering
\setbeamercolor{background canvas}{bg=white}
\setbeamersize{text margin left=1cm,text margin right=1cm}

\title[AI1110  Assignment-11]{ASSIGNMENT-11}
\subtitle{AI1110}
\author[]{MUKUNDA REDDY \\ AI21BTECH11021}
\date

\begin{document}
  \begin{frame}
      \titlepage
  \end{frame}
  
  \begin{frame}{Outline}
      \tableofcontents
  \end{frame}
  
  \section{Question}
  \begin{frame}{Exercise 12-3}
      Show that if X(t) is normal with $\eta_x = 0$ and
      $R_x (\tau) = 0$ for $|\tau| > a$, then it is 
      correlation-ergodic.
  \end{frame}
  
  \section{Solution}
  \begin{frame}{Solution}
      We know that the process x(t) is covariance-ergodic if and only if
      $$ \frac{1}{T} \int_{0} ^{T} C_{zz} (\tau) d\tau 
         \xrightarrow[T\rightarrow \infty] \ 0 $$
         if x(t) is a normal process then equation simplifies to
         \begin{equation}
         \label{eq1}
             C_{zz} (\tau) = C(\lambda + \tau)C(\lambda - \tau) + C^{2}(\tau)
         \end{equation}
  \end{frame}
  
  \begin{frame}{Solution}
   Here we can replace
  \begin{align*}
      C(\lambda + \tau) &= R_x (\lambda + \tau) \\
      C(\lambda - \tau) &= R_x (\lambda - \tau)  \\
      C^{2}(\tau)   &=  R_x ^{2}(\tau) \\
  \end{align*}
      \implies C_{zz} (\tau) = R_x (\lambda + \tau)R_x (\lambda - \tau) + R_x ^{2}(\tau) ,\: \: z(t) = x(t+\lambda)x(t)
  \end{frame}
  
  \begin{frame}{Solution}
      It is given that $R_x (\tau) = 0$ for $|\tau| > a$,
      Substituting the values we get $C_{zz} (\tau) = 0$
      for $|\tau| > \lambda + a$. Froim the above integral if we substitute integral it tends to 0.
     $$ \frac{1}{T} \int_{0} ^{T} C_{zz} (\tau) d\tau =0$$
     so from the if and only if condition x(t) is
     correlation-ergodic.
  \end{frame}
  
  \end{document}